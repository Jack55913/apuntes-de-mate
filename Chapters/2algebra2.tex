\part{Segundo semestre}
\chapterimage{2.pdf}
\chapter{Álgebra II}


\section{Función lineal} %(1.5 semanas)
\subsection{Concepto general de función.}
\subsection{Variación directamente proporcional}
\subsection{Función Lineal}
\subsection{Tabular y graficar funciones lineales.}








\section{Sistema de Ecuaciones lineales} %(4 semanas)
\subsection{Conceptos básicos.}
\subsubsection{Concepto de sistema de ecuaciones.}
\subsubsection{Concepto de solución de un sistema de ecuaciones lineales.}
\subsubsection{Concepto de solución consistente, inconsistente y dependiente.}
\subsection{Operaciones y procedimientos.}
\subsubsection{Métodos de solución de sistemas.}
\subsubsection{Igualación.}
\subsubsection{Sustitución.}
\subsubsection{Reducción (eliminación, sumas o restas).}
\subsubsection{Determinantes.}
\subsubsection{Gráfica de sistema de ecuaciones lineales de dos variables.}
\subsection{Aplicaciones y problemas.}
\subsubsection{Plantear y resolver problemas.}
\subsubsection{Empleo de un sistema de ecuaciones lineales.}







\section{Ecuaciones cuadráticas} %(3 semanas.)
\subsection{Ecuaciones cuadráticas y sus raíces o soluciones}
\subsection{La propiedad del producto cero y la resolución de ecuaciones por factorización.}
\subsection{Solución de ecuaciones cuadráticas completando el trinomio cuadrado perfecto.}
\subsection{Resolución de ecuaciones cuadráticas por la fórmula general.}
\subsection{El discriminante de una ecuación cuadrática y el tipo de raíces.}
\subsection{Solución de ecuaciones reducibles a cuadráticas, mediante un cambio de variable,}
\subsection{Resolución de ecuaciones que contienen radicales.}







\section{Funciones y desigualdades cuadráticas} %(4 semanas)
\subsection{Funciones cuadráticas.}
\subsection{Formas de representar una función cuadrática: tabular, gráfica y mediante una expresión algebraica}
\subsection{Elementos de una función cuadrática: intersección con los ejes coordenados; vértice de la parábola (máximo o mínimo), eje de simetría.}
\subsection{Transformaciones de la gráfica de y= x2}
\subsubsection{$y= (x+h)^2$}
\subsubsection{$y= ax^2$}
\subsubsection{$y=ax^2$}
\subsubsection{$y=x^2+k$}
\subsubsection{$y=a(x+h)^2+k$}
\subsection{Desigualdades cuadráticas.}
\subsection{Desigualdades lineales (un repaso)}
\subsection{Métodos para resolver una desigualdad cuadrática: Gráfico y Algebraico.}









\section{Sistema de ecuaciones cuadráticas} %(2 semanas)

\subsection{Ecuaciones cuadráticas de dos variables.}
\subsection{Gráficas de ecuaciones cuadráticas.}
\subsection{Métodos de resolución de sistemas cuadráticos}







\section{Expresiones exponenciales y logarítmicas} %(4 semanas)
\subsection{Exponentes racionales, función exponencial.}
\subsection{Logaritmos y funciones logarítmicas.}
\subsection{Propiedades de los logaritmos.}
\subsection{Logaritmos comunes y naturales.}
\subsection{Ecuaciones exponenciales y logarítmicas.}
\subsection{Problemas de crecimiento y decrecimiento.}
