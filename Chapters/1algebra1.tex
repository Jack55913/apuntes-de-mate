\part{Primer semestre}
\chapterimage{1.pdf}
\chapter{Álgebra I}


\section{SISTEMAS NUMÉRICOS}% (3 semanas)
\subsection{Números Naturales, Enteros, Racionales, Irracionales, Reales y Primos.}
\subsection{Operaciones en los números: enteros y racionales.}
\subsection{Jerarquía de las operaciones y uso de símbolos de agrupación.}
\subsection{Problemas verbales con números racionales.}








\section{OPERATIVIDAD CON POLINOMIOS} %(3 semanas)
\subsection{Lenguaje común a lenguaje algebraico, primera parte.}
\subsection{Expresión y término algebraicos.}
\subsection{Términos semejantes.}
\subsection{Reducción de términos semejantes.}
\subsection{Eliminación de signos de agrupación.}
\subsection{Suma y resta de expresiones algebraicas.}
\subsection{Leyes de los exponentes para multiplicación y división de expresiones algebraicas.}
\subsection{Multiplicación de expresiones algebraicas.}
\subsection{División de expresiones algebraicas.}
\subsection{Lenguaje común a lenguaje algebraico, segunda parte.}
\subsection{Práctica 1. Operatividad con polinomios.}








\section{ECUACIONES DE PRIMER GRADO}% (3 semanas)

\subsection{Conceptos básicos}
\subsubsection{Ecuación}
\subsubsection{Propiedades de la igualdad.}
\subsection{Operaciones y procedimientos }
\subsubsection{Factorización por factor común.}
\subsubsection{Resolución de Ecuaciones de primer grado con una incógnita, con coeficientes enteros, fraccionarios, literales y despejes en formulas.}
\subsection{Problemas y aplicaciones}
\subsubsection{Plantear y resolver ecuaciones lineales a partir de problemas verbales.}


\section{FUNCIÓN LINEAL} % (3 semanas)
\subsection{Concepto general de función.}
\subsection{Variación directamente proporcional}
\subsection{Función Lineal}
\subsection{Tabular y graficar funciones lineales.}
\subsection{Aplicaciones y problemas.}
\subsection{Practica 2. Función lineal.}


\section{SISTEMAS DE ECUACIONES LINEALES}% (3 semanas)
\subsection{Conceptos básicos.}
\subsubsection{Sistema de ecuaciones.}
\subsubsection{Solución de un sistema de ecuaciones lineales.}
\subsubsection{Sistemas consistentes, inconsistentes y dependientes.}

\subsection{Operaciones y procedimientos.}

\subsubsection{Método de Igualación. (2x2 y 3x3)}
\subsubsection{Método de Sustitución. (2x2 y 3x3)}
\subsubsection{Método de Reducción (eliminación, sumas o restas). (2x2 y 3x3)}
\subsubsection{Método de Determinantes. (2x2 y 3x3)}

\subsection{Practica 3. Gráfica de sistema de ecuaciones lineales. (2x2)}

\subsection{Aplicaciones y problemas.}
\subsubsection{Plantear y resolver problemas.}
\subsubsection{Empleo de un sistema de ecuaciones lineales.}





\section{DESIGUALDAD LINEAL} % (1 semana)
\subsection{Conceptos}
\subsubsection{Concepto de intervalo.}
\subsubsection{Representar intervalos en la recta de los reales.}
\subsubsection{Desigualdad.}
\subsubsection{Propiedades de la desigualdad.}
\subsection{Operaciones y procedimientos}
\subsubsection{Resolver desigualdades lineales en forma gráfica y analítica.}

