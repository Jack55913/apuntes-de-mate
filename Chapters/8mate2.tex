\part{Segundo semestre}
\chapterimage{8.pdf}
\chapter{Matemáticas II}
\section{Aplicaciones de la derivada} %8 HORAS


\subsection{Razones de cambio} 
\subsubsection{Cambio instantáneo de  una variable respecto a otra. (derivada puntual)}
\subsubsection{La recta tangente a una curva  como interpretación geomètrica de la razón de cambio instantánea}
\subsubsection{La recta tangente paralela al eje de las abscisas.}





\subsection{Intervalos de crecimiento y decrecimiento de una curva.}
\subsubsection{Intervalos}
\subsubsection{Crecimiento, decrecimiento o constancia de una curva }
\subsubsection{La derivada como criterio de crecimiento o decrecimiento de una curva en cualquiera de sus puntos.}


\subsection{Intervalos de concavidad}
\subsubsection{Concavidad hacia arriba y hacia abajo.}
\subsubsection{Puntos de inflexión}

\subsection{Valores y puntos críticos}
\subsubsection{Definición de valores y puntos críticos.}
\subsubsection{Concepto de máximos mínimos y puntos de inflexión.}
\subsubsection{Graficaciòn de funciones}


\subsection{La derivada en otras áreas del conocimiento.}
\subsubsection{Conceptos de otras áreas del conocimiento, como:}
\subsubsection{La velocidad de un móvil, leyes de la oferta y la demanda, reacción química, fertilización de cultivos.}









\section{La integral indefinida} % 10
\subsection{Concepto de una Primitiva.}
\subsubsection{Definición de la función primitiva o antiderivada.}
\subsubsection{Significado de la constante de integración.}
\subsection{Reglas Básicas de Integración y Cambio de Variable (Regla de la Cadena).}
\subsubsection{Técnicas que  requieren la regla de la cadena para antiderivación y aquellas que implican un cambio de variable.}
\subsection{Uso de las Tablas de  Integración.}
\subsubsection{Fórmulas básicas de integración.}




\section{La integral definida} % 10
\subsection{Problema del cálculo de áreas.}
\subsubsection{Definición de la notación sigma.}
\subsubsection{Propiedades de la suma empleando la notación sigma.}
\subsubsection{El límite de una sucesión.}
\subsubsection{Concepto de suma inferior y superior.}
\subsubsection{Límite de las sumas superior e inferior}
\subsection{Integral Definida}
\subsubsection{Definición de una suma de Riemann}
\subsubsection{Definición de la Integral Definida}
\subsubsection{Propiedades de la Integral Definida}
\subsection{Teorema Fundamental del Cálculo}
\subsubsection{Cambios de variables para integrales definidas.}
\subsubsection{Teorema del valor medio para integrales}
\subsubsection{Valor medio de una función.}
\subsubsection{Primer  teorema fundamental  del  cálculo}
\subsubsection{Segundo Teorema fundamental del cálculo}




\section{Aplicación de la integral definida} % 20 HORAS
\subsection{Área bajo la curva y área entre curvas.}
\subsection{Longitud de una curva.}
\subsubsection{Cálculo de la longitud de una curva.}
\subsection{Volumen de sólidos de revolución}
\subsection{Centros de gravedad e inercia.}
\subsubsection{Relación existente entre el cálculo de volúmenes de sólidos de revolución independientemente del eje de giro.}
\subsubsection{Determinar el centro de gravedad de un sólido.}
\subsubsection{Determinar el momento de inercia de un sólido.}



\section{Técnicas de integración} % 16 HORAS

\subsection{Integración por partes}
\subsection{Integración trigonométrica}
\subsection{Integración por fracciones parciales.}
