\part{Quinto semestre}
\chapterimage{5.pdf}
\chapter{Cálculo Diferencial}



\section{Razones de cambio} % 3 clases
\subsection{Introducción}
\subsection{Razones de cambio y su cuantificación.}
\subsection{Razones de cambio, pendientes y curvas.}
\subsection{Cálculo de razones de cambio instantáneas}



\section{Funciones} % 12 clases
\subsection{Concepto de función. Notación y clasificación:}
\subsubsection{Algebraicas: racionales e irracionales.}
\subsubsection{Trascendentes: trigonométricas, logarítmicas y exponenciales.}
\subsection{Elementos esenciales de una función (dominio y contra dominio)}
\subsection{Evaluación de funciones y Gráficas de funciones: constante, lineal, cuadrática.}
\subsection{Gráficas de funciones algebraicas y trascendentes.}
\subsection{Funciones definidas por intervalos.}
\subsection{Operaciones con funciones (suma, resta, multiplicación, división y composición).}
\subsection{Función inversa.}




\section{Límites y continuidad} % 9 clases
\subsection{Concepto de límite de una función.}
\subsection{Teoremas sobre límites.}
\subsection{Cálculo de límites.}
\subsection{Continuidad de funciones.}






\section{La derivada} % 12 clases
\subsection{Concepto de derivada de una función.}
\subsection{Interpretación geométrica y física de la derivada de una función.}
\subsection{Reglas de derivación de funciones.}
\subsection{Ecuaciones de las rectas tangente y normal a una curva.}
\subsection{Derivadas de funciones implícitas}
\subsection{Derivadas de orden superior.}








\section{Aplicaciones de la derivada} % 12 clases
\subsection{Funciones crecientes y decrecientes}
\subsection{Máximos y mínimos de una función.}
\subsubsection{Definición de puntos críticos.}
\subsubsection{Criterio de la primera derivada para determinar máximos y mínimos.}
\subsection{Concavidad} 
\subsubsection{Definición de concavidad.}
\subsubsection{Determinación de los intervalos de concavidad.}
\subsubsection{Funciones cóncavas hacia arriba y cóncavas hacia abajo.}
\subsubsection{Puntos de inflexión.}
\subsubsection{Criterio de la segunda derivada para la determinación de máximos y mínimos.}
\subsection{Análisis de funciones aplicando la derivada.}
\subsection{Problemas de aplicación.}
\subsubsection{Problemas de razón de cambio instantáneo.}
\subsubsection{Problemas de optimización.}
