\part{Cuarto semestre}
\chapterimage{4.pdf}
\chapter{Geometría Analítica}


\section{Conceptos fundamentales} % (15 horas)
\subsection{Introducción y Plano Cartesiano.}
\subsection{Distancia entre dos puntos.}
\subsection{División de un segmento en una razón dada.}
\subsection{Punto medio de un segmento.}
\subsection{Conceptos de ángulo de inclinación y pendiente.}
\subsection{Ángulo entre dos rectas.}
\subsection{Paralelismo y Perpendicularidad.}
\subsection{Cálculo de Áreas de polígonos.}

\section{Línea recta} %(15 horas)
\subsection{Definición de línea recta como lugar geométrico.}
\subsection{Ecuación de la recta conociendo las coordenadas de un punto localizado en ella y la pendiente de la misma.}
\subsection{Ecuación de la recta dados dos puntos distintos.}
\subsection{Ecuación de la recta dadas su pendiente y su ordenada al origen.}
\subsection{Ecuación simétrica de la recta.}
\subsection{Ecuación general de la recta.}  
\subsection{Distancia de un punto a una recta.}





\section{Circunferencia} %(10.5 horas)
\subsection{Definición de la circunferencia como lugar geométrico.}
\subsection{Ecuación ordinaria de la circunferencia y su gráfica.}
\subsection{Ecuación general de la circunferencia.}
\subsection{Propiedades de la circunferencia.}
\subsection{Ecuación de la circunferencia determinada a partir de condiciones dadas.}






\section{Parábola} %(9 horas)
\subsection{Definición de la parábola como lugar geométrico.}
\subsection{Ecuación ordinaria de la parábola con eje de simetría paralelo a los ejes de coordenadas y su gráfica respectiva.}
\subsection{Ecuación general de la parábola con eje horizontal o vertical.}
\subsection{Ecuación de la parábola determinada a partir de condiciones dadas. }










\section{Lugares geométricos} %(9 horas)
\subsection{Definición}
\subsection{Dada una ecuación obtener el lugar geométrico}
\subsection{Dado el lugar geométrico obtener la ecuación.}
