\part{Sexto semestre}
\chapterimage{1.pdf}
\chapter{Cálculo Integral}


\section{Diferencial de una función} % (4 horas), 3 clases
\subsection{Concepto de diferencial de una función.}
\subsection{Interpretación geométrica en todos los posibles casos de curvas.}
\subsection{La diferencial como valor aproximado.}



\section{La integral indefinida} %(20 horas) 15 clases
\subsection{Concepto de integral indefinida. Propiedades.}
\subsection{Interpretaciones de la integral indefinida: geométrica; física; económica.}
\subsection{Integrales inmediatas (uso de las tablas).}
\subsection{Aplicaciones de las integrales indefinidas.}



\section{Métodos de integración} %(20 horas) 15 clases
\subsection{Integración por cambio de variable}
\subsection{Integración por partes.}
\subsection{Integración por fracciones parciales simples.}
\subsection{Uso de las tablas de integración.}
\subsection{Integrales trigonométricas.}
\subsection{Integración por sustitución trigonométrica.}



\section{La integral definida}%(20 horas) 15 clases
\subsection{La integral definida. Propiedades.}
\subsection{Interpretaciones geométrica y física.}
\subsection{Teorema Fundamental del Cálculo.}
\subsection{Aplicaciones de la integral definida.}
\subsubsection{Área bajo una curva.}
\subsubsection{Área entre dos curvas.}
\subsubsection{Volumen de sólidos de revolución.}
\subsubsection{Longitud de una curva.}
\subsubsection{Integral impropia.}
