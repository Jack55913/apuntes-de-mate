\part{Tercer semestre}
\chapterimage{3.pdf}
\chapter{Geometría y Trigonometría} %( 3 horas)

\section{Conceptos básicos}
\subsection{Bosquejo histórico de la Geometría.}
\subsection{Términos no definidos.}
\subsection{Postulados de la Recta.}
\subsection{Axiomas de la Geometría.}
\subsection{Terminología notación. }


\section{Ángulos} %(6 horas)
\subsection{Definición, clasificación de los ángulos.}
\subsection{Teorema de los ángulos opuestos por el vértice.}
\subsection{Ángulos que se forman entre parejas de rectas cortadas por una transversal.}
\subsection{Problemas relativos a ángulos. }




\section{Paralelismo y perpendicularidad}% (6 horas)

\subsection{Definición: característica del paralelismo y la perpendicularidad.}
\subsection{Postulado del paralelismo.}
\subsection{Teorema fundamental del paralelismo.}
\subsection{Problemas de demostración. }


\section{Triángulos} %(4.5 horas)
\subsection{Definición, clasificación de los triángulos.}
\subsection{Teorema de los ángulos interiores de un triángulo.}
\subsection{Rectas y puntos notables de los triángulos. }


\section{Congruencia}%(6 horas)
\subsection{Concepto de congruencia.}
\subsection{Postulados de congruencia.}
\subsection{Teorema de triángulo isósceles.}
\subsection{Problemas de aplicación}

\section{Semejanza} %(10.5 horas)
\subsection{Razones y Proporciones.}
\subsection{Concepto de Semejanza. Triángulos Semejantes.}
\subsection{Postulado de Semejanza.}
\subsection{Teorema de Pitágoras.}
\subsection{Teorema Fundamental de Proporcionalidad. }

\section{Funciones trigonométricas} %(7.5 horas)

\subsection{Introducción a la trigonometría.}
\subsection{Funciones trigonométricas de un ángulo agudo de un triángulo rectángulo.}
\subsection{Manejo de tablas y/o calculadora.}
\subsection{Triángulos especiales.}
\subsection{Solución de triángulos rectángulos, cálculo de áreas.}
\subsection{Ángulos de elevación y depresión. }


\section{Funciones trigonométricas} %(7.5 horas)
\subsection{Sistema de Coordenadas Rectangulares.}
\subsection{Definir grados y radianes, conversiones de un sistema al otro.}
\subsection{Ángulo en posición normal, concepto de ángulo reducido o de referencia.}
\subsection{Definición de las funciones trigonométricas de un ángulo cualquiera en posición normal.}
\subsection{Signo de las funciones trigonométricas en los cuatro cuadrantes.}
\subsection{Ángulos positivos, ángulos negativos.}
\subsection{Funciones trigonométricas inversas. }

\section{Círculo trigonométrico, graficación de las funciones trigonométricas} %(4.5 horas)
\subsection{El círculo trigonométrico.}
\subsection{Funciones trigonométricas definidas como segmentos rectilíneos.}
\subsection{Variaciones de las funciones trigonométricas con respecto a la variación del argumento en los cuatro cuadrantes.}
\subsection{Graficación de las funciones trigonométricas. }

\section{Triángulos oblicuángulos}%(6 horas)
\subsection{Ley de Senos y Cosenos.}
\subsection{Solución de Triángulos Oblicuángulos.}
\subsection{Áreas de Triángulos Oblicuángulos. }






\section{Identidades trigonométricas}% (6 horas) 
\subsection{Identidades fundamentales}
\subsection{Identidades de sumas y diferencias de ángulos}
\subsection{Ángulos dobles y ángulos mitad}





\section{Ecuaciones trigonométricas} %(4.5 horas)
\subsection{Solución de ecuaciones trigonométricas de primer grado.}
\subsection{Solución de ecuaciones trigonométricas de segundo grado.} 
