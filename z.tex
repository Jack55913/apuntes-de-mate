\begin{problem}[Integración por cambio de variable]
    \begin{equation*}
        \int \sin^3{x}\,dx
    \end{equation*}
    \textit{ Sol. }
    \begin{align*}
        \int \sin^3{x}\,dx
    \end{align*}
    Factorizamos el Seno, de tal forma:
    \begin{align*}
        &\int \sin{x}\cdot \sin^2{x}\,dx  =  \int \sin{x}\cdot \left(1 -\cos^2{x}\right)\\
        &\int \sin {x}\,dx + \int{\cos^2{x}\cdot -\sin{x}}\,dx
    \end{align*}
    Haciendo un cambio de variable, obtenemos:
    \begin{align*}
        &u = \cos{x}\\
        &du = -\sin{x}\,dx 
    \end{align*}
    Sustituyendo valores:
    \begin{align*}
        - \cos{x} + \int u^2\,du = -\cos{x} + \frac{1}{3}\cos^3{x} + c
    \end{align*}
    Solución:
    \begin{equation*}
        \therefore -\cos{x} + \frac{1}{3}\cos^3{x} + c
    \end{equation*}
    
\end{problem}


\begin{problem}[Cambio de variable]
    \begin{equation*}
        \int \cos{x}\,dx
    \end{equation*}
\textit{ Sol. }
\begin{align*}
    &\int \cos{x}\,dx =\int \cos{x} \cdot  \cos^2{x}\,dx\\
    &=\int \cos{x}\left(1 - \sin^2{x}  \right)\,dx\\
    &=\int \cos{x}\, dx - \int \cos{x} \sin^2{x}  
\end{align*}
Aplicando cambio de variable
\begin{align*}
    &u = \sin{x}\\
    &du = \cos{x}\,dx
\end{align*}
Resolviendo: 
\begin{align*}
    &\sin{x} - \int u^2\,du\\
    &= \sin{x} - \frac{u^3}{3} + c \\
    &= \sin {x} - \frac{\sin^3{x}}{3}\, dx 
\end{align*}
\textit{ Sol. }
\begin{equation}
    \therefore \sin {x} - \frac{\sin^3{x}}{3}\, dx 
\end{equation}
\end{problem}


\begin{problem}[Cambio de variable]
    \begin{equation}
        \left(\tan{x}\sec{x}\right)^2\,dx
    \end{equation}
Aplicando leyes de los signos:
\begin{align*}
    \left(\tan{x}\sec{x}\right)^2\,dx = \int \tan^2{x}\sec^2{x}\,dx = \int u^2\,du\\
\end{align*}
aplicando cambio de variable:
\begin{align*}
    &u = \tan{x}\\
    &du = \sec^2{x}\,dx 
\end{align*}

\begin{align*}
    \frac{u^3}{3} + c = \frac{\tan^3{x}}{3} + c 
\end{align*}

\end{problem}

\begin{problem}[Cambio de variable]
    \begin{equation}
        \int \frac{\ln{x}}{x}\,dx 
    \end{equation}
    \textit{ Sol. }

\begin{align*}
    u = ln{x}\\
    du = \frac{1}{x}\,dx
\end{align*}

\begin{align*}
    \int u\,du = \frac{u^2}{2} + c 
\end{align*}
\begin{equation*}
    \therefore = \frac{\ln^2{x}}{2} + c
\end{equation*}
\end{problem}



\begin{problem}[Integración por sustitución]
    \begin{equation}
        \int \left(\sqrt{x} - \sqrt[3]{x} \right)^{3}\, dx
    \end{equation}
    \textit{ Sol. }
Según las leyes de los binomios al cubo, se desarrolla:
\begin{align*}
    &\left(a - b\right)^3 = a^3 - 3a^2b + 3ab^2 - b^3\\
    &\left(\sqrt{x} - \sqrt[3]{x}\right)^3 = x^{\frac{3}{2}} + 3x \cdot x^{\frac{1}{3}} + 3x^{\frac{1}{2}} \cdot x^{\frac{2}{3}} - x\\ 
    &= x^{\frac{3}{2}} + 3x^{\frac{4}{3}} + 3x^{\frac{7}{6}} - x
\end{align*}
Separando en cuatro integrales:
    \begin{align*}
        &I = \int x^{\frac{3}{2}}\,dx + \int 3x^{\frac{4}{3}}\,dx + \int 3x^{\frac{7}{6}}\,dx - \int x\,dx\\
        &\frac{2}{5}x^{\frac{5}{2}} - 3 \left(\frac{3}{7}\right)  + 3\left(\frac{6}{13}\right)x^{\frac{13}{6}} \frac{1}{2}x^2 + c\\
        &= \frac{2}{5} \sqrt{x^5} - \frac{9}{7}\sqrt[3]{x^7} + \frac{18}{13} \sqrt[6]{x^{13}} - \frac{1}{2}\sqrt{x} + c
    \end{align*}
\end{problem}


\begin{problem}[Integración por sustitución]
    \begin{equation}
        \int \left(1 - 3x\right)^{ - \frac{3}{2}} \, dx
    \end{equation}
\end{problem}

\begin{problem}[Integración por partes]
    \begin{equation}
        \int x^2e^{- 2x} \, dx
    \end{equation}
Esta integral es de la forma:
\begin{equation*}
    \int x^2e^{- 2x} \, dx = \int u\,du = uv -\int u\, du
\end{equation*}
haciendo un cambio de variable:
\begin{align*}
    &u = x^2&& du = e^{ - 2}x\,dx\\
    &du = 2x\,dx&& u = - \frac{1}{2}e^{ - 2x}
\end{align*}
Sustituyendo: 
\begin{align*}
    = x^2\left( - \frac{1}{2}e^{ -2x} \right) + \frac{1}{2} \int e^{- 2x}\cdot 2x\,dx\\
    = - \frac{1}{2}x^2e^{ - 2x} +\int x\cdot e^{- 2x}\,dx
\end{align*}
Aplicando un nuevo cambio de variable, obtenemos:
\begin{equation*}
    J = \int x \cdot e^{- 2x}\, dx
\end{equation*}
\begin{align*}
    &u = x&& dv = e^{ - 2x}\, dx\\
    &du = dx&& v = - \frac{1}{2}e^{- 2x}
\end{align*}
Se obtiene la nueva ecuación:
\begin{align*}
    J = \left( - \frac{x}{2}e^{- 2x} \right) + \int \frac{1}{2}e^{ - 2x}\, dx\\
    J = \left( - \frac{x}{2}e^{ - 2x} \right) + \frac{1}{2}\int e^{ - 2x}\, dx\\
    J =\left( - \frac{x}{2}e^{- 2x}\right) + \frac{1}{2}\left(- \frac{1}{2}\right)e^{- 2x} + c
\end{align*}
Volviendo a la I:
\begin{align*}
    = - \frac{1}{2}x^2e^{- 2x} - \frac{x}{2}e^{ - 2x} - \frac{1}{4} e^{ - 2x} + c\\
    \therefore I = -\frac{1}{2}e^{ - 2x}\left(x^2 + x + \frac{1}{2} \right)
\end{align*}
\end{problem}




\begin{problem}[Integración por parciales]
    \begin{equation}
        \int \frac{x^2-2}{x^3 - 4x}\,dx
    \end{equation}

    \textit{ Sol. }

    \begin{align*}
        &\int \frac{x^2-2}{x\left(x^2 - 4\right)}\,dx = \int \frac{x^2 -2}{x\left(x - 2\right)\left(x + 2\right)}\,dx\\
        &x^2- 2 = \frac{A}{x} + \frac{B}{x - 2} + \frac{C}{x + 2}\left(x(x - 2)(x +2)\right)\\
        &x^2 - 2 = A\left( (x - 2)(x +2) \right) + B \left(x(x + 2)\right) + C\left(x(x - 2)\right)\\ 
    \end{align*}
Con x=2
\begin{align*}
    (2)^2 - 2 = B\left(2(2 + 2)\right)
\end{align*}
Entonces 2= 8B y $B= \frac{1}{4}$
\begin{align*}
    (- 2)^2- 2 = C\left(x(x - 2)\right)\\
\end{align*}
Se obtiene que 2=8C y $C=\frac{1}{4}$ pero x=0
\begin{align*}
    - 2 = A(0 - 2)(0 + 2)
\end{align*}
y - 2 = - 4A, entonces $A =\frac{1}{2}$; La integral queda como:
\begin{align*}
    &\int \left( \frac{\frac{1}{2}}{x} + \frac{\frac{1}{4}}{x - 2} +\, \frac{\frac{1}{4}}{x + 2}\right)\\
    &\frac{1}{2 }\int \frac{\frac{1}{2}}{x} + \frac{1}{4}\int \frac{dx}{x - 2} +\, \frac{1}{4}\int \frac{dx}{x + 2}\\
    &\therefore\frac{1}{2}\ln{x} + \frac{1}{4}\ln{x - 2} + \frac{1}{4}\ln{x + 2} + c
\end{align*}
\end{problem}

\begin{problem}[a]
    Área entre la curva $y=\left\lvert 3x-6\right\rvert,x=-1, x=4 $

    \textit{ Sol. }


\end{problem}



\begin{problem}[a]
    \begin{equation}
        I = \int e^x\cos^2{e^x} \cdot \sin{e^x}\,dx 
    \end{equation}
    Aplicando cambio de variable:
    \begin{align*}
        &u = e^x\\ 
        &du = e^x\, dx
    \end{align*}
    Sustituyendo
    \begin{align*}
    I = \int u \cdot \cos^2{u} \sin{u}  \,du\\
    \end{align*}
    Volviendo a hacer un cambio de variable,
\begin{align*}
    &z = \cos{u}\\
    &dz = \sin{u}\, du
\end{align*}
Sustituyendo:
\begin{align*}
    &J = - \int z^2\, dz\\
    &J = - \frac{z^3}{3} + c\\
    &I = \frac{\cos^3{u}}{3} + c \\
    &I = - \frac{\cos^3{e^x}}{3} + c
\end{align*}
\end{problem}






























